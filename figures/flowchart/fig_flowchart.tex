\documentclass[tikz]{standalone}
\usepackage[T1]{fontenc}
\usepackage[utf8]{inputenc}
\usepackage{pgfplots}
\usepackage{grffile}
\pgfplotsset{compat=newest}
\usetikzlibrary{plotmarks,positioning,fit}
\usepgfplotslibrary{patchplots,groupplots}
\usepackage{siunitx}
\usepackage{amsmath}

\newcommand{\myincimg}[1]{\includegraphics[width=3cm]{#1}}

\begin{document}

\begin{tikzpicture}
  \tikzstyle{img}=[%
  draw,
  inner sep=0cm,
  outer sep=0cm,
  minimum width=3cm,
  minimum height=3cm];

  % source images
  \node[img,label={magnitude}] (M3) {\myincimg{src-5.png}};
  \node[img,below left of=M3,node distance=1.5cm] (M2) {\myincimg{src-3.png}};
  \node[img,below left of=M2,node distance=1cm] (M1) {\myincimg{src-1.png}};
  \node[img,right of=M3,node distance=3cm,label={phase}] (P3) {\myincimg{src-6.png}};
  \node[img,below left of=P3,node distance=1.5cm] (P2) {\myincimg{src-4.png}};
  \node[img,below left of=P2,node distance=1cm] (P1) {\myincimg{src-2.png}};
  % \node[draw,fit={(M1) (M3) (P1) (P3)},inner sep=20pt] {};
  \node[draw=none,above left of=P1,font=\tiny,node distance=4.5cm]{TE};
  \draw[->] ([xshift=-0.1cm,yshift=0.1cm]M1.north west) node (a) {} --
  ([xshift=0.1cm,yshift=0.3cm]M2.north west) node[xshift=0.2cm,yshift=0.2cm,rotate=45] (b) {...};
  \node[draw=none,below=0.3cm of P1,anchor=north,text width=6cm,inner sep=0,outer sep=2pt] (SRCtext) {%
    MR scan: TIMGRE sequence
    % \begin{itemize}
    % \item TIMGRE sequence
    % \item + concomitant gradient correction
    % \end{itemize}
  };

  % WFI
  \node[img,right=3cm of P3,label={PDFF}] (pdff) {\myincimg{pdff-1.png}};
  \node[img,anchor=west,label={$R_2^{\ast}$}] (R2s) at (pdff.east) {\myincimg{R2s-1.png}};
  \node[img,anchor=west,label={total field}] (fm) at (R2s.east) {\myincimg{total_field_ppm-1.png}};
  \node[draw=none,below=0.3cm of R2s,anchor=north,text width=8cm,inner sep=0,outer sep=2pt] (WFItext) {%
    Field mapping: region-growing + IDEAL
    % \begin{itemize}
    % \item region-growing IDEAL + Laplacian unwrapping
    % \item graph cut
    % \end{itemize}
  };
  % \node[draw,fit={(fm) (WFItext)},inner sep=2pt] {};

  % BFR
  \node[img,below=3cm of fm,label=local field] (lf) {\myincimg{local_field_ppm-1.png}};
  \node[draw=none,below=0.3cm of lf,anchor=north,text width=6.3cm,inner sep=0,outer sep=2pt] (BFRtext) {%
    Background field removal: \\ Laplacian boundary value method
    % \begin{itemize}
    % \item Laplacian boundary value method
    % \item Projection onto dipole fields
    % \end{itemize}
  };
  % \node[draw,fit={(lf) (BFRtext)},inner sep=20pt] {};

  % QSM
  \node[img,left=6cm of lf,label=susceptibility] (qsm) {\myincimg{chi-1.png}};
  \node[draw=none,below=0.3cm of qsm,anchor=north,text width=8cm,inner sep=0,outer sep=2pt] (QSMtext) {%
    Dipole inversion: \\
    (i) \(\ell_2\)-regularized closed-form solution, (ii) \(\ell_2\)-regularized morphology-enabled dipole inversion conjugate gradient solution, (iii) TV-regularized morphology-enabled dipole inversion Nesterov's algorithm
    % \begin{itemize}     
    % \item \(\ell_2\)-TV closed-form solution
    % \item \(\ell_2\)-TV morphology-enabled dipole inversion conjugate gradient solution
    % \item \(\ell_1\)-TV morphology-enabled dipole inversion Nesterov's algorithm
    % \end{itemize}
  };
  % \node[draw,fit={(qsm) (QSMtext)},inner sep=20pt] {};


  % paths
  \draw[->] (P3.east) -- (pdff.west);
  \node[coordinate,above right=2cm and 1.5cm of lf] (step) {};
  \draw (fm) -| (step);
  \draw[->] (step) |- (lf);
  \draw[->] (lf) -- (qsm);
\end{tikzpicture}


\end{document}